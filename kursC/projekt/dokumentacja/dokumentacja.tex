% Kartka rozmiaru a4, pozioma, dwie kolumny
\documentclass[a4paper,onecolumn, oneside]{article}

% większy odstęp między kolumnami
\setlength{\columnsep}{3cm}

% UTF-8
\usepackage[T1]{fontenc}
\usepackage[utf8]{inputenc}

% Czcionka, która sensownie wygląda na monitorze.
\usepackage{lmodern}

% Przyjazne makra związane z tytułem używanie nazwy autora (\theauthor).
\usepackage{titling}
\setlength{\droptitle}{-2cm}

% Język
\usepackage[polish]{babel}
% \usepackage[english]{babel}

% Rózne przydatne paczki:
% - znaczki matematyczne
\usepackage{amsmath, amsfonts}
% - wcięcie na początku pierwszego akapitu
\usepackage{indentfirst}
% - komenda \url 
\usepackage{hyperref}
% - dołączanie obrazków
\usepackage{graphics}
% - szersza strona
\usepackage[nofoot,hdivide={2cm,*,2cm},vdivide={2cm,*,2cm}]{geometry}

% dane autora
\author{Katarzyna Trzos}

% Tego nie ruszaj
\title{Dokumentacja projektu C}
\date{ \today}

% początek dokumentu
\begin{document}
\maketitle

\section{Wstęp z uzasadnieniem programu i opisem podstawowych celów}

Program otwiera folder ze zdjęciami i tworzy w nim kolejny folder o nazwie "output" w którym znajduje sie plik "hash.txt" z policzonymi hashami w systemie binarnym i w systemie dziesiętnym oraz nazwa pliku(zdjęcia).

\section{ Opis zastosowania programu i sposób użycia czyli dokładne wypunktowanie możliwości programu, znaczenie
argumentów wywołania, klawiszy, etc.}
Podawanie ścieżki do folderu ze zdjęciami, zatwierdzenie ENTER
Wybór oczekiwanego wyjścia (zdjęcia podobne/ duplikaty) strzałki prawo <, lewo >, zatwierdzenie ENTER
 



\section{Opis implementacji w tym krótki opis zawartości modułów, opis fragmentów kodu z referencją na źródła
jeśli zostały zaczerpnięte z przykładów lub napisane na ich bazie. Można używać przenośnych zewnętrznych
bibliotek, które powinny być uzgodnione przy ustalaniu tematu projektu i zawarte także we wstępnym
opisie.}

1. Redukcja rozmiaru. \\
2. Przekonwertowanie na skalę szarości.\\
3. Funkcja hashująca. \\

\section{Wymagania i sposób kompilacji np. skrypt/Makefile/projekt CodeBlocks wymagane wersje zewnętrznych
bibliotek, na jakich systemach powinno działać np. Microsoft/Linux/Mac, na jakich systemach został
program przetestowany: dokładnie w jakiej konfiguracji.}

Program pisany w systemie Linux

Biblioteki zewnętrzne: \\
OpenCV https://opencv.org/ \\
Leptonica http://www.leptonica.org/ \\
GTK https://www.gtk.org/ \\


\section{ Referencje do bibliotek i innych źródeł użytych w trakcie pisania projektu.} 
biblioteki stb_image\\
GTK https://www.gtk.org/ \\


\end{document}

